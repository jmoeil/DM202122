\documentclass[11pt,oneside,a4paper]{article}

\usepackage[T1]{fontenc}
\usepackage[utf8]{inputenc}
\usepackage[french]{babel}
\usepackage{amsmath,amsfonts,amsthm,amssymb}
\usepackage{cancel}
\usepackage{fancyhdr}
\usepackage{lastpage}
\usepackage{bm}
\usepackage{fontspec}
\usepackage{mathrsfs}
\usepackage{calrsfs}
\usepackage{wasysym}
\usepackage{dsfont}
\usepackage{caption}
\usepackage{multirow}
\usepackage{physics}
\usepackage{siunitx}
\usepackage{tikz}
\usepackage{hyperref}
\usepackage{cleveref}
\usepackage{pgfplotstable}
\usepackage{wrapfig}
\usepackage{graphicx}
\usepackage{subfiles}
\usepackage{systeme}
\usepackage{enumitem}
\usepackage{xcolor}
\usepackage{titlesec}
\usepackage{lmodern}
\usepackage{chngcntr}
\usepackage{textcomp}
\usepackage{geometry}
\geometry{top=2cm, bottom=2cm, left=1cm, right=1cm}
\pgfplotsset{width=10cm,compat=1.16}

% Counter stuff
\counterwithin{equation}{section}
\setcounter{tocdepth}{3}

% Random utilities...
\newcommand{\ti}{\ensuremath{\times}}
\newcommand{\h}{\ensuremath{\hbar}}
\newcommand{\med}{\ensuremath\cdot}
\newcommand\au[2]{\left.#1\right|_{#2}}
\renewcommand{\d}{\mathrm{d}}

% Special formatting for paragraphs
\titleformat{\paragraph}
  {\normalfont\normalsize\bfseries}
  {\theparagraph}
  {1em}
  {}
\titlespacing*{\paragraph}
  {0pt}
  {3.25ex plus 1ex minus .2ex}
  {1.5ex plus .2ex}

% Gestion of headers and footers
\pagestyle{fancy}
\renewcommand\headrulewidth{1pt}
\renewcommand\footrulewidth{1pt}
\fancyfoot[L]{MOEIL Juian}
\fancyfoot[C]{Page \thepage/\pageref{LastPage}}
\fancyfoot[R]{\today}

% Allow usage of smileys
\newfontfamily\DejaSans{DejaVu Sans}

% Defining new theorem environments
\newtheorem{theorem}{Théorème}[section]
\newtheorem{definition}[theorem]{Définition}
\newtheorem{lemma}[theorem]{Lemme}
\newtheorem{property}[theorem]{Proposition}
\newtheorem{corollary}[theorem]{Corollaire}
\newtheorem{remark}[theorem]{Remarque}
\newtheorem{reminder}[theorem]{Rappel théorique}
\newtheorem{example}[theorem]{Exemple}

\renewcommand{\proofname}{Preuve}
\renewcommand{\qedsymbol}{\ensuremath{\blacksquare}}

% Setup of hyperref features
\hypersetup{
  colorlinks=true,
  linkcolor=blue,
  urlcolor=purple,
  pdftitle="DM202122 - MOEIL Juian"
}

% Definition boite grise avec couleur grise définie
\definecolor{BGgris}{RGB}{222,230,230}
\newcommand\bg[2]{%
  \begin{center}%
    \fcolorbox{white}{BGgris}{%
      \parbox{.9\linewidth}{%
        \begin{large} \textit{#1} \end{large} \\%
        #2%
      }%
    }%
  \end{center}%
}

% Boite d'alerte
\definecolor{BGorange}{RGB}{255, 216, 154}
\newcommand\probleme[1]{%
  \begin{center}%
    \fcolorbox{white}{BGorange}{%
      \parbox{.9\linewidth}{%
        \begin{large} \textbf{Problème} \end{large} \\%
        #1%
      }%
    }%
  \end{center}%
}

\setlength{\fboxsep}{2em}

\title{%
  \textbf{PHYS-F302 - Mécanique Quantique} \\
  \textit{Devoir maison - Etats cohérents, oscillateur harmonique et décohérence}
}
\author{%
  \href{mailto:juian.moeil@ulb.be}{Moeil Juian}
}
\date{%
  \textbf{Université Libre de Bruxelles} \\
  \emph{Année académique 2021-2022}
}

\begin{document}
\maketitle
\tableofcontents
\newpage

\begin{remark}
  Les termes en $\bm{gras}$ indiquent soit des opérateurs, soit des vecteurs. A comprendre en fonction du contexte. On pose $\doteq$ comme un symbole pour "égal par définition à".
\end{remark}
\section{Partie I - Etude des états cohérents}
On se restreint au cas unidimensionnel, en se munissant des opérateurs position $\bm{X}$ et impulsion $\bm{P}$, satisfaisant la relation de commutation canonique $[\bm{X},\bm{P}] = i\h$.

\subsection*{Introduction des concepts clés}
\begin{definition}[Ecart quadratique moyen]
  Soit $\bm{A}$ une observable. L'écart quadratique moyen de $\bm{A}$ dans l'état $\rho = \ketbra{\psi}$ pour tout $\ket{\psi}\in\mathcal{H}$ est 
  \begin{equation}
    \Delta\bm{A} = \sqrt{\expval{A^2}-\expval{A}^2} = \expval{\left(\bm{A}-\expval{\bm{A}}\right)^2}.
  \end{equation}
\end{definition}

\begin{definition}[Inégalité de Heisenberg]
  Pour toute observable $\bm{A}$ et $\bm{B}$, nous avons que
  \begin{equation}
    \label{Heisenberg inequality}
    \Delta\bm{A}\Delta\bm{B}\geq\frac{1}{2}\norm{\expval{[\bm{A},\bm{B}]}}.
  \end{equation}
\end{definition}

\begin{definition}
  \label{def etat coherent saturation}
    On définit un état cohérent comme un état $\rho = \ketbra{\psi}$ dans lequel l'inégalité de Heisenberg est saturée, c'est à dire tel que
    \begin{equation}
      \Delta\bm{X}\Delta\bm{P} = \frac{\h}{2}.
    \end{equation} 
\end{definition}

En partant de cette définition, nous allons voir qu'il est possible d'obtenir une définition algébrique (généralement plus simple à manipuler) d'un état cohérent. En particulier, l'étude des propriétés des états cohérents nous permettra d'obtenir deux formulations équivalentes; l'une en terme de l'opérateur destruction dans la section \ref{opérateur d'échelle}, et l'autre - dans la section \ref{opérateur déplacement} - en terme de l'opérateur d'échelle. On finira par constater l'existence d'une relation de fermeture en \ref{section relation de fermeture}. Pour faire tous cela, nous avons besoin de l'expression analytique d'un état cohérent : nous consacrons la section \ref{fonction d'onde d'un état cohérent} à la déterminer.

\subsection{Fonction d'onde d'un état cohérent}
\label{fonction d'onde d'un état cohérent}
\begin{property}
  La fonction d'onde d'un état cohérent est donné par
  \begin{align}
    \label{fonction d'onde}
    \psi(x) &= \frac{\exp(-\frac{i}{2\pi}\expval{\bm{X}}\expval{\bm{P}})}{\left(\pi l^2\right)^{1/4}}\exp(-\frac{\left(x-\expval{\bm{X}}\right)^2}{2l^2}+\frac{i\expval{\bm{P}}x}{\hbar}).
  \end{align}
\end{property}

\begin{remark}
  L'expression recherchée dépendra de trois paramètres réels, que nous prendrons dans cette discussion égales à $\expval{\bm{X}},\expval{\bm{P}}$ et $l \doteq \sqrt{2}\Delta\bm{X}$.
\end{remark}

\begin{proof}
\paragraph{Développement général}
Suivons le même raisonnement que celui exploité pour démontrer l'inégalité de Heisenberg dans le cours. On introduit $\mathcal{O}(\lambda) = f^\dagger(\lambda)f(\lambda)$, où $\begin{cases}
  f(\lambda) = \tilde{\bm{A}}+i\lambda\tilde{\bm{B}}\\
  \tilde{\bm{A}} = \bm{A}-\expval{\bm{A}}\\
  \tilde{\bm{B}} = \bm{B}-\expval{\bm{B}}
\end{cases}$.\\

Observons que la valeur moyenne de $\mathcal{O}(\lambda)$ est positive. En effet,
\begin{equation*}
  \forall\ket{\psi}\in H, \quad \expval{f^\dagger(\lambda)f(\lambda)}{\psi} = \norm{f(\lambda)\ket{\psi}}^2 \geq 0.
\end{equation*}

Or, $\expval{\mathcal{O}(\lambda)} = \lambda^2\expval{\tilde{\bm{B}^2}}+\lambda\expval{i[\tilde{\bm{A}},\tilde{\bm{B}}]}+\expval{\tilde{\bm{A}}^2}$. Notons qu'il s'agit d'une équation quadratique en $\lambda$ : ainsi, le minimum doit donc être donné par 
\begin{align}
  \au{\frac{d\expval{\mathcal{O}(\lambda)}}{d\lambda}}{\lambda=\lambda^*} = 0 &= 2\lambda^*\expval{\tilde{\bm{B}}^2}+\expval{i[\tilde{\bm{A}},\tilde{\bm{B}}]}\notag\\
  \lambda^* &= -\frac{\expval{i[\tilde{\bm{A}},\tilde{\bm{B}}]}}{2\expval{\tilde{\bm{B}}^2}} = -\frac{\expval{i[\tilde{\bm{A}},\tilde{\bm{B}}]}}{2\left(\Delta \tilde{\bm{B}}\right)^2}
\end{align}

\paragraph{Application aux opérateurs position et impulsion}
Posons $\bm{A} = \bm{X}$ et $\bm{B} = \bm{P}$. Le minimum, $\lambda^*$, s'exprime alors par $\frac{\h}{2\left(\Delta \bm{P}\right)^2}$. En particulier, nous trouvons alors que $\expval{\mathcal{O}(\lambda)} = 0$ si et seulement si 
\begin{align}
  \left(\tilde{\bm{X}}+i\lambda^*\tilde{\bm{P}}\right)\ket{\psi} = 0\notag
\end{align}
Ce que nous réécrivons comme
\begin{align}
  \left(\bm{X}-\expval{\bm{X}}\right)\ket{\psi} = -\frac{i\h}{2\left(\Delta \bm{P}\right)^2}\left(\bm{P}-\expval{\bm{P}}\right)\ket{\psi}\label{1.5}
\end{align}

Posons $l = 2\Delta X$. Puisqu'un état cohérent sature l'inégalité de Heisenberg, nous pouvons écrire $2\left(\Delta\bm{P}\right)^2 = \frac{\h^2}{2\left(\Delta\bm{X}\right)^2} = \frac{\h^2}{l^2}$, de sorte que \eqref{1.5} se réécrive
\begin{align}
  \left(\bm{X}-\expval{\bm{X}}\right)\ket{\psi} &= -i\frac{l^2}{\h}\left(\bm{P}-\expval{\bm{P}}\right)\label{1.6}
\end{align}
Finalement, en notant que dans la base position l'opérateur impulsion s'écrit $\bm{P}\ket{\psi} = \frac{\h}{i}\frac{\partial}{\partial x}\ket{\psi}$, nous pouvons alors mettre en évidence l'\emph{équation différentielle}
\begin{align}
  -x\psi(x)+\expval{\bm{X}}\psi(x)&=\frac{il^2}{\h}\left(\frac{\h}{i}\psi'(x)-\expval{\bm{P}}\psi(x)\right)\notag\\
  \psi'(x) &= \left[\frac{\expval{\bm{X}}-x}{l^2}+\frac{i}{\h}\expval{\bm{P}}\right]\psi(x).\notag
\end{align}
Que nous résolvons par séparation des variables. Cela donne:
\begin{align}
  \ln(\frac{\psi}{\psi_0}) &= \left[\frac{\left(\bm{X}-\frac{1}{2}x\right)x}{l^2}+\frac{i}{\h}\expval{\bm{P}}x\right]   &\; \forall\psi_0\in\mathbb{C}\notag\\
  \ln(\frac{\psi}{\psi_0}) &= \left[-\frac{\left(x-\expval{\bm{X}}\right)^2}{2l^2}+\frac{\expval{\bm{X}}}{2l}+\frac{i}{\h}\expval{\bm{P}}x\right]\notag\\
  \psi(x) &= \psi_0\exp(-\frac{\left(x-\expval{\bm{X}}\right)^2}{2l^2}+\frac{i}{\h}\expval{\bm{P}}x)
\end{align}
La fonction d'onde d'un état cohérent est donc donnée par une \emph{gaussienne} !

\paragraph{Constante de normalisation}
Il ne nous reste plus qu'à déterminer la valeur de $\psi_0\in\mathbb{R}$. Pour ce faire, nous exploitons la propriété suivante

\begin{property}
  \label{prop:condition de normalisation anal}
    Pour toute fonction d'onde $\Psi(x,t)$, la condition de normalisation s'écrit 
    \begin{equation}
        \int_{-\infty}^{+\infty}dx \; \norm{\Psi(x,t)}^2= 1.
    \end{equation}
\end{property}
\begin{proof}
  Il s'agit d'une condition imposée par les postulats de la mécanique quantique. Nous la prenons donc pour acquise et ne devons pas la prouver.
\end{proof}
Commencons par réarranger les termes dans l'exponentielle
\begin{align}
  \psi(x) &= \psi_0 \; \exp(-\frac{x^2}{2l^2}-\frac{\expval{X}^2}{2l^2}+x\left(\frac{\expval{X}}{l^2}+\frac{i\expval{P}}{\h}\right))\notag
\end{align}
La condition de normalisation nous indique alors que
\begin{align}
  \norm{\psi_0}^2 \exp(-\frac{\expval{X}^2}{l^2})\int_{-\infty}^{+\infty} dx \; \exp(-\left(\frac{x}{l}\right)^2+\frac{2\expval{X}}{l^2}x) &= 1\notag\\
  \norm{\psi_0}^2 \sqrt{\pi l^2} &= 1 &\psi_0 = \frac{1}{\left(\pi l^2\right)^{1/4}}\in\mathbb{R}\notag
\end{align} 
Puisque le choix de phase globale n'a aucune importance physique, nous avons la liberté de prendre une phase de notre choix. En particulier, nous pouvons donc écrire 
\begin{align}
  \psi(x) &= \frac{\exp(-\frac{i}{2\pi}\expval{X}\expval{P})}{\left(\pi l^2\right)^{1/4}}\exp(-\frac{\left(x-\expval{X}\right)^2}{2l^2}+\frac{i\expval{P}x}{\hbar})\notag
\end{align}
ce qui conclut notre discussion sur la fonction d'onde d'un état cohérent. \end{proof}
\subsection{Opérateurs d'échelle} \label{opérateur d'échelle}
Introduisons l'opérateur
\begin{equation}
  \bm{a} = \frac{1}{\sqrt{2}}\left(\frac{\bm{X}}{l}+\frac{il\bm{P}}{\h}\right).
\end{equation}
$\bm{a}$ porte le nom d'\emph{opérateur destruction}. Nous dressons un rappel-issu du cours théorique- de ses propriétés dans la base de Fock en \ref{sec:fock vs. coherent}: cela suffira à comprendre la logique derrière son nom.
\subsubsection{Commutateur $[a,a^\dagger]$}
L'expression de commutation des bosons est donnée par
\begin{equation}
    [\bm{a},\bm{a}^\dagger] = \frac{1}{2} [\frac{\bm{X}}{l}+\frac{il\bm{P}}{\hbar},\frac{\bm{X}}{l}-\frac{il\bm{P}}{\hbar}] = -\frac{i}{2\h} [\bm{X},\bm{P}]-\frac{i}{2\h}[\bm{X},\bm{P}] = -\frac{i}{\h}[\bm{X},\bm{P}] = \mathbb{I},
\end{equation}
en exploitant directement la définition de $\bm{a}$.

\subsubsection{Valeur propre de $a$ - résolution analytique}
\label{resolution analytique}

Placons nous dans un espace de Hilbert, et prenons un vecteur $\ket{\psi}$ lui appartenant. Nous recherchons les solutions de l'équation aux valeurs propres
\begin{align}
  \bm{a}\ket{\alpha} &= \alpha\ket{\alpha} = \frac{1}{\sqrt{2}}\left(\frac{\bm{X}}{l}+\frac{il\bm{P}}{\h}\right)\ket{\alpha}.
\end{align}
Commencons par remarquer que l'expression précédente nous permet de directement déduire la relation $(3)$ de l'énoncé:
\begin{align}
  \expval{\bm{a}}{\alpha} &= \alpha\braket{\alpha} = \frac{1}{\sqrt{2}}\left(\frac{\expval{\bm{X}}}{l}+\frac{il\expval{\bm{P}}}{\h}\right).
\end{align}

Le problème se réduit donc à résoudre
\begin{align}
  \frac{1}{\sqrt{2}}\left(\frac{\bm{X}}{l}+\frac{il\bm{P}}{\h}\right)\ket{\alpha} &= \frac{1}{\sqrt{2}}\left(\frac{\expval{X}}{l}+\frac{il\expval{P}}{\h}\right)\ket{\alpha}\\
  \left(\bm{X}-\expval{\bm{X}}\right)\ket{\alpha} &= -\frac{il^2}{\h}\left(\bm{P}-\expval{\bm{P}}\right)\ket{\alpha}.
\end{align}
Or, il s'agit de l'équation \eqref{1.6}, que nous avons déjà entièrement résolu en $\ref{fonction d'onde d'un état cohérent}$. Nous voyons en particulier que, à $l$ fixé, la solution est unique. Nous pouvons donc écrire $\ket{\psi} = \ket{\alpha}$. La fonction d'onde $\braket{x}{\psi}=\braket{x}{\alpha}$ est déterminée par la gaussienne que nous avons trouvé en \ref{fonction d'onde d'un état cohérent}.\\

Nous obtenons alors le résultat qui suit ; lequel nous sera utile dans les discussions qui suivent. 
\begin{theorem}
  $\ket{\psi}$ est un état cohérent (tel que défini en \ref{def etat coherent saturation}) si et seulement si $\ket{\psi}$ vérifie l'équation aux valeurs propres
  \begin{align}
    \label{etat coherent}
    \bm{a}\ket{\psi} &= \alpha\ket{\psi} &\forall\; \alpha\in\mathbb{C}
  \end{align}
  c'est à dire si et seulement si $\ket{\psi}$ est un état propre de l'opérateur annihilation $\bm{a}$.
\end{theorem}

\begin{proof}
  La preuve dans le sens direct est le développement débutant au début de \ref{resolution analytique}. Nous ne faisons pas la preuve dans le sens réciproque ($\bm{a}\ket{\psi}=\alpha\ket{\psi}$ pour tout $\alpha\in\mathbb{C}$ implique que $\braket{x}{\psi}=\psi(x)$ est de la forme \ref{fonction d'onde}), car cela ne fait pas l'objet de ce devoir.
\end{proof}

\begin{remark}
  Observons que, comme annoncé, nous venons de mettre en évidence une formulation équivalente à \ref{def etat coherent saturation} d'un état cohérent, en exploitant l'opérateur annihilation.
\end{remark}

Nous avons à présent les connaissances pour démontrer la proposition suivante. Elle sera particulièrement utile dans la suite.

\begin{property}
  \label{coherent in fock}
  Soit $\left\{\ket{n} : n\in\mathbb{N}\right\}$ une base de Fock, et soit $\left\{\ket{\alpha} : \alpha\in\mathbb{C}\right\}$ un ensemble d'états cohérents. Alors, pour tout $\alpha\in\mathbb{C}$, $\ket{\alpha} = e^{-\norm{\alpha}^2/2}\sum_n \frac{\alpha^n}{\sqrt{n!}}\ket{n}$.
\end{property}

\begin{proof}
  Puisque la base de Fock et la base des états cohérents peuvent tous les deux être utilisés pour décrire un Oscillateur Harmonique, il doit être possible d'exprimer tout vecteur dans l'une des bases en terme de l'autre. En particulier, cela se traduit par $\ket{\alpha} = \sum_n c_n \ket{n}$, pour tout $c_n\in\mathbb{C}$. En exploitant \eqref{etat coherent}, nous trouvons
  \begin{align}
    \bm{a}\ket{\alpha} = \sum_n c_n a\ket{n} = \sum_n c_n\sqrt{n}\ket{n-1} &= \alpha\sum_n c_n \ket{n}
  \end{align}
  De cette égalité, nous déduisons $c_n\sqrt{n}=\alpha c_{n-1}$. Par récurrence, nous pouvons alors montrer que $c_n = \frac{\alpha^n}{\sqrt{n!}}c_0$. En injectant cela dans notre somme, on se retrouve avec $\ket{\alpha} = c_0\sum_n \frac{\alpha^n}{\sqrt{n!}}\ket{n}$. Par normalisation,
  \begin{align}
    1 = \braket{\alpha} &= \norm{c_0}^2 \sum_{n,m\geq 0} \frac{\bar{\alpha}^m\alpha^n}{\sqrt{n!m!}}\braket{m}{n} = \norm{c_0}^2\sum_{n\geq 0} \frac{\left(\norm{\alpha}^2\right)^n}{n!} = \norm{c_0}^2e^{\norm{\alpha}^2}
  \end{align}
  où nous avons utilisé la représentation sous forme de série de la fonction exponentielle. En particulier, nous avons alors $c_0 = e^{-\frac{1}{2}\norm{\alpha}^2}$, ce qui implique la relation recherchée.
\end{proof}

\subsubsection{Produit scalaire avec le vide}
\label{intégrale gaussienne}

La solution est triviale en exploitant la propriété \ref{coherent in fock}, que nous venons de démontrer. En effet,
\begin{align}
  \braket{0}{\alpha} &= e^{-\frac{1}{2}\norm{\alpha}^2}\sum_{n\geq 0}\frac{\alpha^n}{\sqrt{n!}}\braket{0}{n} = e^{-\frac{1}{2}\norm{\alpha^2}}
\end{align}
où nous avons exploité le fait que $\braket{0}{n}$ est nul pour tout $n\neq 0$.

\subsubsection{`Fock state' vs. `Coherent state'}\label{sec:fock vs. coherent}
Dans le cours théorique, nous avons introduit la base de Fock comme étant l'ensemble des vecteurs $\ket{n}$ diagonalisant l'opérateur nombre $N=\bm{a}^\dagger \bm{a}$. Nous avons en particulier démontré une série de propriétés remarquables sur ces vecteurs propres $\ket{n}$, en voici quelques unes.

\begin{lemma}
  L'opérateur nombre $\bm{N}$ est hermitien : $\bm{N}^\dagger = \bm{a}^\dagger \left(\bm{a}^\dagger\right)^\dagger$.
\end{lemma}

\begin{property}
  L'opérateur nombre $\bm{N}$ est positif, dans le sens défini au cours : tout vecteur propre $\ket{n}$ de $\bm{N}$ est associé à une valeur propre $n\geq 0$.
\end{property}
\begin{lemma}
  $[\bm{N},a]=-a$ et $[\bm{N},a^\dagger] = a^\dagger$
\end{lemma}
\begin{property}
  \label{Vp destruction}
  Soit $\ket{\varphi}$ un vecteur propre de $\bm{N}$ de valeur propre $\nu$ : $\bm{N}\ket{\varphi} = \nu\ket{\varphi}$. Alors,
  \begin{itemize}
      \item $\bm{a}\ket{\varphi}$ est vecteur propre de $\bm{N}$ de valeur propre $\nu-1$.
      \item Si $\nu = 0$, alors $\bm{a}\ket{\varphi} = 0$ : cela correspond à l'état fondamental.
  \end{itemize}
\end{property}
Cette propriété nous assure que si nous trouvons un vecteur propre $\ket{n}$ de valeur propre $n$ sur $\bm{N}$, alors nous pouvons construire un vecteur propre linéairement indépendant $\bm{a}\ket{n}$ tel que $n-1$ soit une valeur propre de $\bm{N}$.
\begin{remark}
  Cela justifie le nom de l'opérateur $\bm{a}$ : l'opérateur annihilation.
\end{remark}
\begin{property}
  Soit $\ket{\varphi}$ un vecteur propre de $\bm{N}$ de valeur propre $\nu$ : $\bm{N}\ket{\varphi} = \nu\ket{\varphi}$. Alors,
  \begin{itemize}
      \item $\bm{a}^\dagger\ket{\varphi}$ est non nul.
      \item $\bm{a}^\dagger\ket{\varphi}$ est un vecteur propre de valeur propre $\nu+1$.
  \end{itemize}
\end{property}
Cette propriété nous assure que si nous trouvons un vecteur propre $\ket{n}$ de valeur propre $n$ sur $\bm{N}$, alors nous pouvons construire un vecteur propre linéairement indépendant $\bm{a}^\dagger\ket{n}$ tel que $n+1$ soit une valeur propre de $\bm{N}$.
\begin{remark}
  Cela justifie le nom que porte l'opérateur $\bm{a}^\dagger$ : l'opérateur création.
\end{remark}
Finalement, ces propriétés nous permettent de voir le résultat suivant.
\begin{theorem}
  Soit l'opérateur nombre $\bm{N} = \bm{a}^\dagger\bm{a}$. Le spectre de $\bm{N}$ est un sous-ensemble des naturels : $Spectre \bm{N}\subseteq\mathbb{N}$.
\end{theorem}


De ce théorème, nous tirons notre interprétation des états $\ket{n}$. Il s'agit de l'ensemble des vecteurs - linéairement indépendant les uns des autres - diagonalisant l'opérateur $\bm{N}$. Ces états forment une base - la \emph{base de Fock}, orthonormée. 

Cette dernière base est bien à distinguer de la base des états cohérents : en effet, si $\ket{n}$ est un élément de l'espace de Fock, rien n'assure que $\ket{n}$ diagonalise l'opérateur annihilation $\bm{a}$. De même, si $\ket{\alpha}$ est un état cohérent, $\bm{a}^\dagger \bm{a}\ket{\alpha} = \alpha \bm{\alpha}^\dagger\ket{\alpha}$ ne donne aucune information sur les états propres de $\bm{N}$. A la différence des états propres de $\bm{N}$ - décrivant un nombre fixe d'excitations - les états $\ket{\alpha}$ en possèdent un nombre indéterminé: elles se dotent cependant d'une \emph{phase fixe}. Nous avons bien deux bases distinctent, donnant des informations différentes sur les opérateurs introduit par la solution de Dirac de l'Oscillateur Harmonique.\\

Maintenant que la différence entre les deux bases est claire, attelons-nous à montrer ce qui les relie. Introduisons la relation $\ket{n} = \frac{\left(\bm{a}^\dagger\right)^n}{\sqrt{n!}}\ket{0}$. Notons que tout état cohérent $\ket{\alpha}$ peut se réécrire sous la forme $\sum_n \ket{n}\braket{n}{\alpha}$, où nous avons utilisé la relation de fermeture des états de Focks. En particulier, cela nous incline à déterminer la valeur du produit scalaire $\braket{n}{\alpha}$. Rappelons la relation $\bm{a}^\dagger\ket{n} = \sqrt{n+1}\ket{n+1}$, dont l'adjointe est $\bra{n}\bm{a} = \bra{n+1}\sqrt{n+1}$.
\begin{align}
  \bra{n}a\ket{\alpha} = \alpha\braket{n}{\alpha} &= \sqrt{n+1}\braket{n+1}{\alpha}\notag
\end{align}
En remplacant $n$ par $n-1$, nous obtenons
\begin{align}
  \alpha\braket{n-1}{\alpha} &= \sqrt{n}\braket{n}{\alpha}\notag\\
  \frac{\alpha}{\sqrt{n}}\braket{n-1}{\alpha} &= \braket{n}{\alpha}\notag
\end{align}
Soit, par récurrence, 
\begin{align}
  \braket{n}{\alpha} &= \frac{\alpha^n}{\sqrt{n!}}\braket{0}{\alpha} = e^{-\frac{1}{2}\norm{\alpha}^2}\frac{\alpha^n}{\sqrt{n!}}\label{eq:scalar fock vs coherent}
\end{align}
Nous continuerons ce raisonnement dans la section \ref{sec:vp algebrique}.
\subsubsection{Note sur la non-orthogonalité de la base des états cohérents}
Nous pouvons voir directement, en exploitant la proposition \ref{coherent in fock}, que la base des états cohérents n'est \textbf{pas} orthonormale. En effet, pour tout $\alpha,\beta$ sur le plan complexe
\begin{align}
  \braket{\beta}{\alpha} &= e^{-\frac{1}{2}\norm{\beta}^2}e^{-\frac{1}{2}\norm{\alpha}^2}\sum_{n,m\geq 0}\frac{\left(\beta^*\right)^m}{\sqrt{m!}}\frac{\left(\alpha\right)^n}{\sqrt{n!}}\braket{m}{n}
  = e^{-\frac{1}{2}\norm{\beta}^2}e^{-\frac{1}{2}\norm{\alpha}^2} \sum_n \frac{\left(\beta^*\alpha\right)^n}{n!} = e^{-\frac{1}{2}\norm{\beta}^2}e^{-\frac{1}{2}\norm{\alpha}^2}e^{\beta^*\alpha}\label{15}
\end{align}
où nous avons à nouveau utilisé la représentation de la fonction exponentielle en terme de série. Observons que 
\begin{align}
  -\frac{1}{2}\norm{\beta}^2-\frac{1}{2}\norm{\alpha}^2+\beta^*\alpha &= -\frac{1}{2}\beta^*\beta - \frac{1}{2}\alpha^*\alpha+\beta^*\alpha\\
  &= -\frac{1}{2}\left(\beta^*\beta+\alpha^*\alpha-\beta^*\alpha-\alpha^*\beta-\beta^*\alpha+\alpha^*\beta\right)\\
  &= -\frac{1}{2}\norm{\alpha-\beta}^2-\frac{1}{2}\left(\alpha^*\beta-\beta^*\alpha\right)
\end{align}
En particulier, notons que $\norm{\alpha-\beta}^2$ est un nombre réel, tandis que $\alpha^*\beta-\beta^*\alpha$ est bien complexe. 

\begin{remark}
  Nous exploiterons ce résultat dans le cadre de la relation de complétude des états cohérents dans la section \ref{section relation de fermeture}.
\end{remark}

Nous pouvons alors réécrire \eqref{15} sous la forme
\begin{align}
  \braket{m}{n} &= e^{-\frac{1}{2}\norm{\alpha-\beta}^2}e^{-\frac{1}{2}\left(\alpha^*\beta-\beta^*\alpha\right)} = e^{-\frac{1}{2}\norm{\alpha-\beta}^2}e^{-iIm\left\{\alpha^*\beta\right\}}.
\end{align}
Cela nous donne une condition de \emph{presqu'orthogonalité}.
\begin{align}
  \braket{m}{n} &\approx 0
\end{align}
lorsque $\norm{\alpha-\beta}^2$ >> $1$.

\subsubsection{Valeur propre de $a$ - résolution algébrique}\label{sec:vp algebrique}
Reprenons le raisonnement développé dans le cadre de la section \ref{sec:fock vs. coherent}. En particulier, rappelons que tout état cohérent peut se réécrire avec la relation de fermeture $\sum_n \ket{n}\braket{n}{\alpha}$. Le produit scalaire entre un état cohérent et un élément de la base de Fock est donné par \eqref{eq:scalar fock vs coherent}, et nous permet donc d'écrire
\begin{align}
  \ket{\alpha} = e^{-\frac{1}{2}\norm{\alpha}^2}\sum_{n\geq 0}\frac{\alpha^n}{\sqrt{n!}}\ket{n} = e^{-\frac{1}{2}\norm{\alpha}^2}\sum_{n\geq 0}\frac{\left(\alpha\bm{a}^\dagger\right)^n}{n!}\ket{0} = e^{-\frac{1}{2}\norm{\alpha}^2+\alpha\bm{a}^\dagger}\ket{0}
\end{align}
On voit donc qu'un état cohérent peut-être \textit{généré} à partir du vide. 
\subsubsection{Interprétation en terme de particules}

Pour terminer cette section sur les opérateurs d'échelle, posons nous quelques questions statistiques sur l'état $\ket{\alpha}$, en interprétant cette fois les opérateurs $\bm{a}^\dagger$ et $\bm{a}$ en terme de particules. En effet, nous pouvons voir $\ket{n}$ comme un système contenant n particules d'énergie $\h\omega$ n'intéragissant pas entre-eux. Dans cette vision, l'opérateur annihilation $\bm{a}^\dagger$ peut-être vu comme traduisant le retrait d'une particule dans le système. L'opérateur création $\bm{a}$ caractérise alors l'ajout d'une particule dans le système. Par exemple, l'opérateur $\bm{a}^\dagger\bm{a}\bm{a}^\dagger\bm{a}$ traduit la création de deux particules à la suite de la destruction (l'annihilation) de deux autres.\\

La probabilité de trouver $n$ particules dans le système dans l'état $\ket{n}$ est donnée par la règle de Born.
\begin{align}
  P(n) = \norm{\braket{n}{\alpha}}^2 = \frac{\left(\norm{\alpha}^{2}\right)^ne^{-\norm{\alpha}^2}}{n!}
\end{align}
Il s'agit d'une \emph{distribution de probabilité de Poisson}, de paramètre $\lambda = \norm{\alpha}^2$.\\

Similairement, le nombre moyen de particules dans l'état $\ket{\alpha}$ s'exprime par la relation
\begin{align}
  \expval{\bm{N}} = \norm{\alpha}^2\label{eq:mean N}
\end{align}

\subsection{Opérateur déplacement} \label{opérateur déplacement}
Introduisons, pour tout nombre complexe $\alpha$, l'opérateur déplacement, noté $\bm{D}(\alpha)$, défini par la relation
\begin{align}
  \label{eq:operateur deplacement}
  \bm{D}(\alpha) = e^{\alpha \bm{a}^\dagger-\alpha^*\bm{a}}.
\end{align}

\begin{property}[Relation de Glaubert]
  \label{prop Glaubert}
  Soit $\bm{A}$ et $\bm{B}$ deux opérateurs. L'égalité
  \begin{align}
    \label[]{Glaubert}
    e^{\bm{A}}e^{\bm{B}} &= e^{\bm{A}+\bm{B}}e^{-\frac{1}{2}\left[\bm{A},\bm{B}\right]}
  \end{align}
  tient si et seulement si les opérateurs commutent avec $\left[\bm{A},\bm{B}\right]$.
\end{property}

\begin{lemma}
  \label{lemme displacement}
  L'opérateur déplacement peut - de manière tout à fais équivalente - être défini par la relation
  \begin{align}
    \bm{D}(\alpha) = e^{-\frac{1}{2}\norm{\alpha}^2\mathbb{I}}e^{\alpha\bm{a}^\dagger}e^{-\alpha^*\bm{a}}
  \end{align}
\end{lemma}
\begin{proof}
  Il suffit de remarquer que le commutateur $\left[\alpha\bm{a}^\dagger,-\alpha^*\bm{a}\right] = \norm{\alpha}^2\mathbb{I} \neq 0$ : en particulier, nous avons alors que tout opérateur $\bm{A}$ commute avec $\left[\alpha\bm{a}^\dagger,\alpha^\alpha\bm{a}\right]$, car tout opérateur commute avec l'opérateur identité. Les hypothèses de \ref{prop Glaubert} sont alors respectées : il suffit d'isoler l'exponentielle de la somme d'opérateurs dans \eqref{Glaubert} pour obtenir la relation recherchée.
\end{proof}

\begin{property}
  Nous pouvons similairement définir un état cohérent comme un état généré par l'opérateur déplacement $\bm{D}(\alpha)$ appliqué à l'état du vide $\ket{0}$.
\end{property}
\begin{proof}
  \begin{align}
    \bm{D}(\alpha)\ket{0} &=  e^{-\frac{1}{2}\norm{\alpha}^2}e^{\alpha\bm{a}^\dagger}e^{-\alpha^*\bm{a}}\ket{0} = e^{-\frac{1}{2}\norm{\alpha}^2}e^{\alpha\bm{a}^\dagger} \sum_k \left(-1\right)^k \frac{\bm{a}^k}{k!}\ket{0}
  \end{align}
  Or, nous avons que $\sum_k \bm{a}^k\ket{0} = \bm{a}^0\ket{0} = \ket{0}$.
  \begin{align}
    \bm{D}(\alpha)\ket{0} &= e^{-\frac{1}{2}\norm{\alpha}^2}e^{\alpha\bm{a}^\dagger}\ket{0} =  e^{-\frac{1}{2}\norm{\alpha}^2} \sum_i \frac{\alpha^i}{i!}\left(\bm{a}^\dagger\right)^i\ket{0} = e^{-\frac{1}{2}\norm{\alpha}^2} \sum_i \frac{\alpha^i}{i!}\ket{i} = \ket{\alpha}
  \end{align}
  Ce qui met en lumière la logique derrière le nom de cet opérateur déplacement $\bm{D}(\alpha)$.
\end{proof}

\begin{theorem}
  $\ket{\alpha}$ est un état cohérent si et seulement si $\ket{\psi}$ vérifie la relation
  \begin{equation*}
    \ket{\alpha}=\bm{D}(\alpha)\ket{0}
  \end{equation*}
  c'est à dire si et seulement si l'état cohérent est généré par le vide à travers l'opérateur déplacement.
\end{theorem}

\subsubsection{Unitarité de l'opérateur déplacement}

\begin{definition}
  Un opérateur $\bm{O}$ est unitaire si et seulement si $\bm{O}^\dagger\bm{O} = \mathbb{I}$, c'est à dire si et seulement si $\bm{O}^\dagger = \bm{O}^{-1}$.
\end{definition}

Observons que $\bm{D}(\alpha)\bm{D}(-\alpha) = \mathbb{I}$. Montrer que $\bm{D}(\alpha)$ est un opérateur unitaire revient dès lors à montrer que $\bm{D}^\dagger(\alpha) = \bm{D}(-\alpha)$.
\begin{align}
  \bm{D}^\dagger(\alpha) &= e^{\alpha^*\bm{a}-\alpha\bm{a}^\dagger} = \bm{D}(-\alpha) 
\end{align}

\subsubsection{Déplacement + Déplacement = Déplacement ?}
\emph{Note : Je répond aux questions 3.b et 3.c ici.}

Considérons l'opérateur déplacement de paramètre $\alpha+\beta$, pour tout $\alpha,\beta\in\mathbb{C}$. Celui-ci vaut, par définition,
\begin{align}
  \bm{D}(\alpha+\beta) &= \exp(\alpha\bm{a}^\dagger-\alpha^*\bm{a}+\beta\bm{a}^\dagger-\beta^*\bm{a})
\end{align}
Ce qui revient à faire la somme de deux opérateurs ; soit $\bm{A} = \alpha\bm{a}^\dagger-\alpha^*\bm{a}$ et $\bm{B} = \beta\bm{a}^\dagger-\beta^*\bm{a}$.
\begin{align}
  \left[\bm{A},\bm{B}\right] &= \left[\alpha\bm{a}^\dagger-\alpha^*\bm{a},\beta\bm{a}^\dagger-\beta^*\bm{a}\right] = \alpha^*\beta-\alpha\beta^* = 2i Im(\alpha^*\beta)\mathbb{I}
\end{align}

En utilisant le même argument que dans la preuve du lemme \ref{lemme displacement}, nous avons donc que ($\bm{A},\bm{B}$) commutent avec leur commutateur. Alors, \eqref{Glaubert} nous autorise à écrire
\begin{align}
  \bm{D}(\alpha+\beta) &= e^{\alpha\bm{a}^\dagger-\alpha^*\bm{a}}e^{\beta\bm{a}^\dagger-\beta^*\bm{a}}e^{iIm(\alpha^*\beta)} = \bm{D}(\alpha)\bm{D}(\beta)e^{iIm(\alpha^*\beta)}.
\end{align}
En particulier, en appliquant l'état du vide $\ket{0}$ sur $\bm{D}(\alpha+\beta)$, il est clair que
\begin{align}
  \bm{D}(\alpha+\beta)\ket{0} &= \ket{\alpha+\beta}\notag\\
  e^{iIm(\alpha^*\beta)}\bm{D}(\alpha)\bm{D}(\beta)\ket{0} &= \ket{\alpha+\beta}\notag\\
  \bm{D}(\alpha)\ket{\beta} &= e^{-Im(\alpha^*\beta)}\ket{\alpha+\beta}\label{eq:deplacement sur etat coherent}
\end{align}
\begin{remark}
  Le facteur $e^{\pm iIm(\alpha^*\beta)}$ est une phase globale : il n'a donc aucune importance physique ! 
\end{remark}
\subsection{Relation de fermeture}
\label{section relation de fermeture}
\begin{property}
  La base des états cohérents $\left\{\ket{\alpha} : \alpha\in\mathbb{C}\right\}$ admet la relation de fermeture
  \begin{align}
    \frac{1}{\pi}\iint dRe(\alpha)dIm(\alpha) \; \ket{\alpha}\bra{\alpha} &= \mathbb{I}
  \end{align}
\end{property}
\begin{proof}
  Adoptons la notation $d^2\alpha = dRe(\alpha)dIm(\alpha)$. En vertue de la propriété \ref{coherent in fock}, nous avons
  \begin{align}
    \frac{1}{\pi}\int d^2\alpha \; \ket{\alpha}\bra{\alpha} &= \frac{1}{\pi}\sum_{n,m\geq 0} \frac{\ket{m}\bra{n}}{\sqrt{n!m!}} \left[\int d^2\alpha \; e^{-\norm{\alpha}^{2}} \bar{\alpha}^m\alpha^n\right]
  \end{align}
  Passant en coordonnées polaires, en posant $\alpha=re^{i\theta}$ et $\alpha = rdrd\theta$.
  \begin{align}
    \frac{1}{\pi}\int d^2\alpha \; \ket{\alpha}\bra{\alpha} &= \frac{1}{\pi}\sum_{n,m\geq 0}\frac{\ket{m}\bra{n}}{\sqrt{n!m!}}\left[\int_0^{+\infty}dr\; r^{m+n+1}e^{-r^2} \; \int_0^{2\pi}d\theta \; e^{i\left(n-m\right)\theta}\right]
  \end{align}
  Or, il se trouve que\footnote{En effet, lorsque $n=m$, l'intégrant vaut 1. Lorsque $n\neq m$, nous avons l'intégrale de la combinaison linéaire de deux fonctions périodique sur une période ; ce qui est égal à 0.}
  \begin{align}
    \int_0^{2\pi}d\theta \; e^{i\left(n-m\right)\theta} &= 2\pi\delta_{n,m}
  \end{align}

Dès lors, l'intégrale se réduit à l'expression
\begin{align}
  \frac{1}{\pi}\int d^2\alpha \; \ketbra{\alpha} &= 2\sum_n \frac{\ketbra{n}}{n!}\left[\int_0^{+\infty}dr \; r^{2n+1}e^{-r^2} \right]
\end{align}

En effectuant le changement de variable $s = r^2$ (soit donc $ds = 2r \; dr$), il est claire que nous nous réduisons à $\Gamma(n+1) = n!$

\begin{align}
  \frac{1}{\pi}\int d^2\alpha \; \ket{\alpha}\bra{\alpha} &= \sum_n \ketbra{n} = \mathbb{I}
\end{align}
Ce qui conclut notre preuve.
\begin{verbatim}
  \end{proof}
\end{verbatim}
\end{proof}
\subsubsection{Commentaire sur l'indépendance linéaire}
\emph{Nous pouvons prendre cette partie comme le commentaire demandé à l'exercice 2.e.}

Tout vecteur $\ket{\psi}$ dans la base de Hilbert (dans laquelle se trouve la base des états cohérents) peut donc s'écrire
\begin{align}
  \ket{\psi} = \frac{1}{\pi}\int d^2\alpha \ket{\alpha}\braket{\alpha}{\psi}
\end{align} 
En particulier, si $\ket{\psi} = \ket{\beta}$ est dans la base des états cohérents, 
\begin{align}
  \ket{\beta} &= \frac{1}{\pi} \int d^2\alpha \ket{\alpha}\braket{\alpha}{\beta} = \frac{1}{\pi} \int d^2\alpha \ket{\alpha} \exp(-\frac{1}{2}\norm{\alpha-\beta}^2-\frac{1}{2}\left(\alpha^*\beta-\beta^*\alpha\right))
\end{align}
où nous avons utilisé la relation de non-orthogonalité des états cohérents. Cette équation montre que \emph{les états cohérents ne sont pas linéairement indépendants} : tout état cohérent $\ket{\alpha}$ peut s'écrire en terme d'un autre.

\begin{remark}
  Le terme "base des états cohérents" est donc un abus de language! Nous continuerons à l'utiliser\footnote{Surtout parceque la suite est déjà rédigée ...}, mais il faut garder en tête que les états cohérents ne forment \textbf{pas} une base.
\end{remark}

\bg{\textbf{Résumé}}{
  Nous avons défini trois formulations équivalentes des états cohérents. Celles-ci sont données par les relations suivantes: 
  \begin{align*}
    \psi(x) &= \frac{\exp(-\frac{i}{2\pi}\expval{\bm{X}}\expval{\bm{P}})}{\left(\pi l^2\right)^{1/4}}\exp(-\frac{\left(x-\expval{\bm{X}}\right)^2}{2l^2}+\frac{i\expval{\bm{P}}x}{\hbar}) \quad  \text{(Fonction d'onde d'un état cohérent)}\\
    \bm{a}\ket{\alpha}&=\alpha\ket{\alpha}\quad\text{(Valeur propre de l'opérateur destruction)}\\
    \ket{\alpha}&=\bm{D}(\alpha)\ket{0}\quad\text{(Généré par l'opérateur déplacement)}
  \end{align*}
  Nous exploiterons principalement les deux dernières relations dans la suite de ce devoir.
  }

\section{Partie II - Oscillateur hamonique}
\label{part 2}
Considérons l'Hamiltonien classique d'un oscillateur harmonique, 
\begin{align}
  \label{Hamiltonien OH}
  H &= \frac{P^2}{2m} + \frac{m}{2}\left(\omega X\right)^2
\end{align}
Posons $l = \sqrt{\frac{\h}{m\omega}}$. En introduisant l'opérateur nombre $N = a^\dagger a$, où $a$ est l'opérateur destruction étudié précédemment. On définit les opérateurs position et impulsion de sorte à nous retrouver avec le système
\begin{equation}
  \begin{cases}
    X = \sqrt{\frac{\h}{2m\omega}}\left(a+a^\dagger\right)\\
    P = -i\sqrt{\frac{m\h\omega}{2}}\left(a-a^\dagger\right)
  \end{cases}
  \label{P and X}
\end{equation}
Nous pouvons alors réécrire l'Hamiltonien sous la \emph{forme normale}, soit 
\begin{align}
  \label{Hamiltonien OH normale}
  H = \h\omega\left(N+\frac{1}{2}\mathbb{I}\right).
\end{align}
En particulier, sous cette forme, il est évident que l'énergie est quantifiée et de valeur $E_n = \h\omega\left(n+\frac{1}{2}\right)$.
\subsection{Evolution temporelle}
\subsubsection{Evolution temporelle de l'état cohérent d'un oscillateur harmonique}
Selon le postulat d'évolution, à tout état $\ket{\psi}$ peut-être associé un opérateur hermitien $H$, appelé Hamiltonien, régissant l'évolution temporelle du vecteur d'état au travers de l'équation de Schrödinger.
\begin{equation}
  i\h\frac{d}{dt}\ket{\psi(t,\bm{r})} = H\ket{\psi (t,\bm{r})}
\end{equation}
En particulier, nous avons donc que $\ket{\psi(t,\bm{r})} = \bm{U}(t,t_0)\ket{\psi(t_0,\bm{r})}$, où $\bm{U}(t,t_0)$ est un opérateur unitaire, appelé opérateur d'évolution. Il peut toujours être écrit sous la forme $\bm{U}(t,t_0) = e^{-\frac{iH}{\h}(t-t_0)}$.\\

Supposons que l'oscillateur harmonique est initialement dans l'état cohérent $\ket{\psi (t_0 = 0)} = \ket{\psi_0} = \ket{\alpha}$. \emph{Dans quel état sera-t-il en un temps ultérieur}, $t>t_0 = 0$ ?
\begin{align}
  \ket{\psi(t)} &= e^{-\frac{iHt}{\h}}\ket{\psi_0} = e^{-\frac{iHt}{\h}}\ket{\alpha}  
\end{align}
Bien que cette équation soit vraie, elle ne nous apporte pas vraiment de connaissances utiles: nous voulons exprimer $\ket{\psi} (t)$ dans la base de Fock $\left\{\ket{n} : n\in\mathbb{N}\right\}$. Nous aurons ainsi accès à la base diagonalisant l'Hamiltonien, de sorte que ce dernier se simplifie en
\begin{align}
  \ket{\psi(t)} = e^{-\frac{iHt}{\h}}\ket{\alpha} &= e^{-i\frac{\omega t}{2}}e^{-\frac{1}{2}\norm{\alpha}^2}\sum_{n\geq 0}e^{-i\omega tn}\frac{\alpha^n}{\sqrt{n!}}\ket{n}
  = e^{-i\frac{\omega t}{2}}e^{-\frac{1}{2}\norm{\alpha}^2}\sum_{n\geq 0}\frac{\left(\alpha e^{-i\omega t}\right)^n}{\sqrt{n!}}\ket{n}
  = e^{-i\frac{\omega t}{2}}\ket{\alpha e^{-i\omega t}}\label{psi t}
\end{align}
où nous avons utilisé la proposition \eqref{coherent in fock}. Nous pouvons réécrire \eqref{psi t} sous la forme, plus propre:
\begin{align}
  \ket{\psi(t)} &= \ket{\alpha(t)} &\text{Où }\ket{\alpha(t)} = e^{-\frac{\norm{\alpha}^2}{2}}\left(\sum_{n\geq 0}\frac{\alpha^n}{\sqrt{n!}}e^{-i\left(\frac{1}{2}+n    \right)\omega t}\ket{n}\right) 
\end{align}

Observons que $\ket{\alpha(t)}$ est normalisé. En effet,
\begin{align}
  \braket{\alpha(t)} = e^{-\norm{\alpha}^2}\left(\sum_{n,m\geq 0}\frac{\bar{\alpha}^m\alpha^n}{\sqrt{n!m!}}e^{-i\omega t\left(n-m\right)}\braket{m}{n}\right)
  &= e^{-\norm{\alpha}^2}\left(\sum_{n\geq 0}\frac{\left(\norm{\alpha}^n\right)^2}{n!}\right)\\
  &= e^{-\norm{\alpha}^2}\left(\sum_{n\geq 0}\frac{\left(\norm{\alpha}^2\right)^n}{n!}\right) = e^{-\norm{\alpha}^2}e^{\norm{\alpha}^2} = 1.
\end{align}
où nous avons utilisé la représentation en terme de série de la fonction exponentielle.\\

Déterminons déjà la valeur de $a\ket{\alpha(t)}$, afin de faciliter les calculs dans la suite. 

\begin{align}
  a\ket{\alpha(t)} = e^{-\frac{\norm{\alpha}^2}{2}}\left(\sum_{n\geq 0}\frac{\alpha^n}{\sqrt{n!}}e^{-i\left(\frac{1}{2}+n\right)\omega t}a\ket{n}\right)
  &= e^{-\frac{\norm{\alpha}^2}{2}}\left(\sum_{n\geq 0}\frac{\alpha^n}{\sqrt{n!}}e^{-i\left(\frac{1}{2}+n    \right)\omega t}\sqrt{n}\ket{n-1}\right)\\
  &= \alpha e^{-i\omega t}e^{-\frac{\norm{\alpha}^2}{2}}\left(\sum_{n\geq 0}\frac{\alpha^{n-1}}{\sqrt{(n-1)!}}e^{-i\left(\frac{1}{2}+n-1    \right)\omega t}\ket{n-1}\right)
\end{align}

Le terme en $n = 0$ n'a pas de sens: on pose alors $m=n-1$, afin de pouvoir réécrire la somme en démarrant en $m = 0$.

\begin{align}
  a\ket{\alpha(t)} &= \alpha e^{-i\omega t}e^{-\frac{\norm{\alpha}^2}{2}}\left(\sum_{m\geq 0}\frac{\alpha^{m}}{\sqrt{(m)!}}e^{-i\left(\frac{1}{2}+m    \right)\omega t}\ket{m}\right) = \alpha e^{-i\omega t}\ket{\alpha(t)}
\end{align}

Cette relation nous indique directement la relation correspondante dans la base duale associée. 
\begin{align}
  \bra{\alpha(t)}a^\dagger &= \bra{\alpha(t)}\alpha^*e^{i\omega t}
\end{align}

\begin{definition}
  La moyenne $\expval{A}$ d'une observable $A$ par rapport à un état $\ket{\psi}$ est donnée par
  \begin{align}
    \expval{A} &= \sum_n a_nP(a_n) = \sum_n a_n \expval{P_n}{\psi} = \expval{\sum_n a_nP_n}{\psi} &\expval{A} = \expval{A}{\psi} 
  \end{align}
\end{definition}

\begin{property}
  L'écart quadratique moyen d'une obserable $A$ dans l'état $\ket{\psi}$ est donnée par
  \begin{align}
    \label{ecart quadratique moyen}
    \Delta A = \sqrt{\expval{A^2}{\psi} - \expval{A}{\psi}^2}
  \end{align}
\end{property}

\subsubsection{Evolution temporelle de l'opérateur position}
\label{Evolution temporelle de l'opérateur position}

\paragraph{Moyenne de la position}

La moyenne dans l'état $\ket{\psi(t)}$ est donnée par la relation
\begin{align}
  \expval{X}{{\alpha(t)}} &= \sqrt{\frac{\h}{2m\omega}}\left[\expval{a}{\alpha(t)}+\expval{a^\dagger}{\alpha(t)}\right]\\
  &= \sqrt{\frac{\h}{2m\omega}} \left[\alpha e^{-i\omega t}+\alpha^*e^{i\omega t}\right]\braket{\alpha(t)}\\
  &= \sqrt{\frac{\h}{2m\omega}} \left[\alpha e^{-i\omega t}+\alpha^*e^{i\omega t}\right]
\end{align}

On peut poser $\alpha = \norm{\alpha}e^{i\varphi}$ : l'expression de la moyenne de l'opérateur position $X$ dans l'état $\ket{\alpha(t)}$ se simplifie alors grandement.
\begin{align}
  \label{mean position}
  \expval{X}(t)&= \sqrt{\frac{2\h}{m\omega}}\norm{\alpha}\cos(\omega t-\varphi)
\end{align}

\paragraph{Ecart quadratique moyen de la position}

Du système \eqref{P and X} nous trouvons que 
\begin{align}
  X^2 &= \frac{\h}{2m\omega}\left(a^2+\left(a^\dagger\right)^2+2a^\dagger a+1\right)
\end{align}
où nous avons utiliser la relation de commutation des bosons pour mettre l'équation sous \emph{forme normale}. Nous pouvons alors déterminer la moyenne de $X^2$.
\begin{align}
  \expval{X^2}{\alpha(t)} &= \frac{\h}{2m\omega} \left(\expval{a^2}{\alpha(t)}+\expval{\left(a^\dagger\right)^2}{\alpha(t)}+2\expval{a^\dagger a}{\alpha(t)}+\braket{\alpha(t)}\right)\\
  &= \frac{\h}{2m\omega}\left[\left(\alpha e^{-i\omega t}\right)^2+\left(\alpha^*e^{i\omega t}\right)^2 +2\norm{\alpha}^2+1\right]\braket{\alpha(t)}\\
  &= \frac{\h}{2m\omega}\left[\left(\alpha e^{-i\omega t}+\alpha^*e^{i\omega t}\right)^2+1\right] = \frac{\h}{2m\omega}\left[\norm{\alpha}^2\left(e^{i\varphi}e^{-i\omega t}+e^{-i\varphi}e^{i\omega t}\right)^2+1\right]\label{Mean X^2}
\end{align}
Où nous avons posé $\alpha = \norm{\alpha}e^{i\varphi}$. Nous trouvons finalement que 
\begin{align}
  \expval{X^2}(t) &= \frac{\h}{2m\omega}\left[4\norm{\alpha}^2\cos^2(\omega t-\varphi)+1\right]
\end{align}

En vertue de \eqref{ecart quadratique moyen}, nous avons alors que l'incertitude sur la position est donnée par
\begin{align}
  \Delta X &= \sqrt{\frac{\h}{2m\omega}\left[4\norm{\alpha}^2\cos^2(\omega t-\theta)+1\right] - \frac{2\h}{m\omega}\norm{\alpha}^2\cos^2(\omega t-\theta)} = \sqrt{\frac{\h}{2m\omega}}
\end{align}

\begin{remark}
  Puisque nous avons défini un état cohérent comme un état saturant l'inégalité de Heisenberg, on s'attend à ce que $\Delta P$ soit tel que $\Delta X\Delta P = \frac{\h}{2}$, c'est à dire que $\Delta P = \sqrt{\frac{m\h\omega}{2}}$.
\end{remark}

\subsubsection{Evolution temporelle de l'opérateur impulsion}
\label{Evolution temporelle de l'opérateur impulsion}

Nous adaptons ici les calculs effectués en \ref{Evolution temporelle de l'opérateur position}, pour l'opérateur impulsion.

\paragraph{Moyenne de l'impulsion}

La moyenne dans l'état $\ket{\alpha(t)}$ est donnée par la relation
\begin{align}
  \expval{P}{\alpha(t)} &= -i\sqrt{\frac{m\h\omega}{2}}\left(\expval{a}{\alpha(t)}-\expval{a^\dagger}{\alpha(t)}\right)\\
  &= -i\sqrt{\frac{m\h\omega}{2}}\left(\alpha e^{-i\omega t}-\alpha^*e^{i\omega t}\right)\label{45}
\end{align}
En posant $\alpha = \norm{\alpha}e^{i\varphi}$, nous pouvons simplifier \eqref{45} : de la sorte, la moyenne de l'impulsion est donnée par une expression similaire à \eqref{mean position}.
\begin{align}
  \expval{P}{\alpha(t)} &= -i\sqrt{\frac{m\h\omega}{2}}\norm{\alpha}\left(e^{i\varphi}e^{-i\omega t}-e^{-i\varphi}e^{i\omega t}\right)\\
  &= -i\sqrt{\frac{m\h\omega}{2}}\norm{\alpha} \left(-2i\sin\omega t-\varphi\right)\\
  \expval{P}(t) &= -\sqrt{2m\h\omega}\norm{\alpha}\sin(\omega t-\varphi) \label{mean P}
\end{align} 

\paragraph{Ecart quadratique moyen de l'impulsion}

Du système $\eqref{P and X}$, nous apprenons que 
\begin{align}
  P^2 &= -\frac{m\h\omega}{2}\left(\left(a^\dagger\right)^2+a^2-2a^\dagger a-1\right)
\end{align}
où nous avons exprimer l'égalité sous la \emph{forme normale}. Nous pouvons alors déterminer la moyenne de $P^2$, ce qui donne, sans surprise, un résultat tout à fais similaire à \eqref{Mean X^2}.
\begin{align}
  \label{Mean P^2}
  \expval{P^2}(t)&= \frac{m\h\omega}{2}\left[4\norm{\alpha}^2\sin^2(\omega t-\varphi)+1\right]
\end{align}
L'incertitude sur $P$ s'ensuit.
\begin{align}
  \Delta P &= \sqrt{\frac{m\h\omega}{2}\left[4\norm{\alpha}^2\sin^2(\omega t-\varphi)+1\right] - 2m\h\omega\norm{\alpha}^2\sin^2(\omega t-\varphi)} = \sqrt{\frac{m\h\omega}{2}}
\end{align}

\begin{remark}
  L'incertitude de Heisenberg est bien saturée! Hallelujah!
\end{remark}

Avant de passer à la suite, quelques observations/commentaires sur les différentes quantités obtenue:
\begin{itemize}
  \item Les moyennes des opérateurs - et de leur carré - dépendent explicitement, périodiquement, du temps.
  \item L'incertitude sur ces mêmes opérateurs, par contre, est indépendante du temps: à tout instant donné, l'incertitude sur la mesure de X (resp. de P) dans l'état $\ket{\alpha(t)}$ est la même.
  \item La moyenne de la position \eqref{mean position} et la moyenne de l'impulsion \eqref{mean P} de l'oscillateur harmonique quantique dans un état cohérent oscillent \emph{de la même manière} que dans un oscillateur harmonique classique ! On voit donc ici la véritable puissance de la notion d'état cohérent.
\end{itemize} 

\begin{remark}
  Attention cependant, l'état $\ket{\psi(t)} = \ket{\alpha(t)}$ sera, après une mesure, en vertue des postulats, projeté sur le sous-espace propre associé au résultat de la mesure. Les résultats que nous avons développé ici ne seront alors plus pertinents pour toute mesure ultérieure à la mesure.
\end{remark}

\subsubsection{Evolution temporelle de l'Hamiltonien}

En vertue de l'expression de l'Hamiltonien d'un Oscillateur Harmonique, le calcul de la moyenne se réduit à calculer la moyenne de $P^2$ et de $X^2$, ce qui a été fais en \ref{Evolution temporelle de l'opérateur position} et en \ref{Evolution temporelle de l'opérateur impulsion}. En particulier, le problème de la moyenne de l'Hamiltonien se réduit donc à injecter \eqref{Mean X^2} et \eqref{Mean P^2} dans \eqref{Hamiltonien OH}.
\begin{align}
  \expval{H}{\alpha(t)} &= \frac{1}{2m}\left(\expval{P^2}{\alpha(t)}\right) + \frac{m\omega^2}{2}\expval{X^2}{\alpha(t)}\\
  &= \frac{1}{2m} \left[\frac{m\h\omega}{2}\left(4\norm{\alpha}^2\sin^2(\omega t-\varphi)+1\right)\right] + \frac{m\omega^2}{2}\left[\frac{\h}{2m\omega}\left(4\norm{\alpha}^2\cos^2(\omega t-\varphi)+1\right)\right]\\
  \expval{H} &= \frac{\h\omega}{2}\left(2\norm{\alpha}^2+1\right)\label{mean H}
\end{align}
\begin{remark}
  Observons directement que, contrairement à la moyenne de la position et de l'impulsion, la moyenne de l'Hamiltonien semble être indépendant du temps. Cela fait sens: l'interprétation de l'Hamiltonien d'un système est celui de l'énergie totale présent dans celui-ci. Or, pour un système isolé - tel notre oscillateur harmonique - l'énergie totale doit être conservé, c'est à dire $\dot{H} = 0$.
\end{remark}

Pour calculer l'incertitude sur $H$, il y a essentiellement deux méthodes: la première (la méthode longue) consiste à déterminer $H^2$ en utilisant \eqref{Hamiltonien OH}, ce qui reviendrait à faire le calcul de la moyenne de $X^4,P^4$, etc. La seconde méthode, plus courte, revient à déterminer $H^2$ en exploitant son expression en terme des bosons. Appliquons cette dernière méthode:
\begin{align}
  H^2 = \left(\h\omega\right)^2\left[a^\dagger a+\frac{1}{2}\mathbb{I}\right]^2 &= \left(\h\omega\right)^2\left[\left(a^\dagger a\right)^2+a^\dagger a+\frac{1}{4}\right]
\end{align}

La moyenne de $H^2$ est alors donné par:
\begin{align}
  \expval{H^2}{\alpha(t)} &= \h^2\omega^2\left[\expval{a^\dagger a a^\dagger a}{\alpha(t)}+\expval{a^\dagger a}{\alpha(t)}+\frac{1}{4}\right]\\
  &= \h^2\omega^2\left[\norm{\alpha}^2\expval{aa^\dagger}{\alpha(t)}+\norm{\alpha}^2+\frac{1}{4}\right]\\
  &= \h^2\omega^2\left[\norm{\alpha}^2\expval{\mathbb{I}+a^\dagger a}{\alpha(t)}+\norm{\alpha}^2+\frac{1}{4}\right]\\
  \expval{H^2}&= \h^2\omega^2\left[\norm{\alpha}^4+2\norm{\alpha}^2+\frac{1}{4}\right]
\end{align}
Sans surprise cette fois, la moyenne de $H^2$ est indépendante du temps. Pour finir, nous pouvons déterminer la valeur de l'incertitude de l'Hamiltonien selon la méthode usuelle:
\begin{align}
  \label{delta H}
  \Delta H = \sqrt{\h^2\omega^2\left[\norm{\alpha}^4+2\norm{\alpha}^2+\frac{1}{4}\right]-\frac{\h^2\omega^2}{4}\left(4\norm{\alpha}^4+4\norm{\alpha}^2+1\right)} = \h\omega\norm{\alpha}
\end{align}

\subsection{Comparaison avec un oscillateur harmonique classique}

Nous voulons maintenant comprendre le lien entre la théorie classique de l'oscillateur harmonique et sa théorie quantique. En particulier, pour ce faire, considérons un oscillateur macroscopique correspondant à un pendule, de masse $m=1\si{kg}$ et de longueur $L = 10\si{cm}$. On suppose que ce pendule effectue des petites oscillations autour de sa position d'équilibre, d'amplitude $L\absolutevalue{\theta_{max}}=1\si{cm}$.

De manière générale, un système possédant une énergie potentielle $V(x)$ peut-être approximée en $x=x_0$ par la série de Taylor
\begin{align}
  V(x) = V(x_0) + \left(x-x_0\right)\left.\frac{dV}{dx}\right|_{x_0} + \frac{\left(x-x_0\right)^2}{2!} \left.\frac{d^2V}{dx^2}\right|_{x_0} + \mathcal O(x-x_0)^3
\end{align} 

Le système tendra à tourner autour de la configuration minimisant $V(x)$ - or, par définition, il s'agit de l'endroit où $\frac{dV}{dx}$ disparaît. Dès lors, nous avons que $V(x) = \frac{\left(x-x_0\right)^2}{2} \left.\frac{d^2V}{dx^2}\right|_{x_0} + \mathcal O(x-x_0)^3 \approx \frac{1}{2}k\left(x-x_0\right)^2$. Il s'agit du potentiel d'un oscillateur harmonique pour des petites oscillations autour de $x_0$.

Analysons le système décrit au premier paragraphe en exploitant la formulation Lagrangienne de la mécanique. Ce devoir ne traitant pas de mécanique classique, nous passerons les détails de calcul pour se focaliser sur l'interprétation physique. Placons-nous dans un champ gravitationel $\bm{g} = -g\bm{\bm{y}}$. La Lagrangien du système est 
\begin{equation}
  L = \frac{1}{2}mL^2\dot{\theta}^2 + mg\cos\theta 
\end{equation}
L'équation d'Euler-Lagrange permet de pleinement résoudre cette équation, et nous donne le résultat $\ddot{\theta} + \frac{g}{L}\sin\theta =  0$. En considérant de faibles oscillations autour de la position d'équilibre, nous pouvons admettre l'approximation de MacLaurin $\sin\theta=\theta$, de sorte que nous nous retrouvions avec l'équation différentielle $\ddot{\theta} + \frac{g}{L}\theta = 0$. La solution générale d'une telle équation est donnée par
\begin{align}
  \label{single pendulum}
  \theta(t) = A\cos(\sqrt{\frac{g}{L}}t+\varphi)
\end{align}
où $A$ est l'amplitude et $\varphi$ est une phase quelconque ; ces deux constantes dépendant des conditions initiales.\\

\emph{Un état cohérent imite donc intimement l'évolution d'un oscillateur harmonique classique.} En particulier, en mettant en parallèle les équations \eqref{mean position} et \eqref{single pendulum}, nous trouvons les relations suivantes.
\begin{equation}
  \omega = \sqrt{\frac{g}{L}}\qquad
  l = \sqrt{\frac{\h}{m\omega}} = \sqrt{\frac{L\h}{mg}}\qquad
  A = \sqrt{2}l\norm{\alpha} = \sqrt{\frac{2L\h}{mg}}\norm{\alpha} \label{eq:46}
\end{equation}
Or, nous connaissons la valeur de $A$ pour notre OH classique! Elle est donnée par $A = L\absolutevalue{\theta_{max}} = 1\si{cm}$. En isolant le terme $\norm{\alpha}$ dans l'expression de $A$, nous avons alors que
\begin{equation}
  \label{89}
  \norm{\alpha}= A\sqrt{\frac{mg}{2L\h}}
\end{equation}
pour l'état cohérent reproduisant le plus fidèlement le mouvement du pendule.\\

Finalement, en associant les résultats \eqref{mean H}, \eqref{delta H} et \eqref{89}, nous trouvons
\begin{equation}
  \frac{\Delta H}{\expval{H}} = \frac{2\norm{\alpha}}{2\norm{\alpha}^2+1} = \frac{A\sqrt{2mgL\h}}{mgA^2+L\h}
\end{equation}
Application numérique: FLEMME

\newpage
\section{Partie III - Décohérence}
\label{partie 3}
Dans cette partie, nous voulons étudier un modèle très simple de la décohérence quantique. Considérons un oscilliateur harmonique "macroscopique"\footnote{Un oscillateur dont l'état initial est un état cohérent (ou une superposition d'états cohérents), caractérisé  par un nombre de particules $\expval{\bm{N}}$ très grand.}. On modélise l'environnement de notre oscillateur harmonique par un autre oscillateur harmonique - de même fréquence. L'Hamiltonien d'un tel système est donné par
\begin{equation}
  \label{eq:Hamiltonien partie III}
  H = \h\omega\left(\bm{a}^\dagger\bm{a}+1/2\right)+\h\omega\left(\bm{b}^\dagger\bm{b}+1/2\right)+\h\lambda\left(\bm{a}\bm{b}^\dagger+\bm{a}^\dagger\bm{b}\right)
\end{equation}
Dans cette expression, le premier terme correspond à l'oscillateur harmonique étudié, le second terme modélise l'oscillateur harmonique de l'environnement et le troisième terme décrit le couplage entre l'environnement et l'oscillateur. On pose $\lambda\in\mathbb{R}^+$.

On se place dans l'espace de Hilbert $\mathcal{H} = \mathcal{H}_a\otimes\mathcal{H}_b$. On suppose qu'une base de $\mathcal{H}_a$ est donné par $\ket{n,a} = \frac{\left(\bm{a}^\dagger\right)^n}{\sqrt{n!}}\ket{0,a}$ et qu'une base de $\mathcal{H}_b = \frac{\left(\bm{b}\right)^m}{\sqrt{m!}}\ket{0,b}$. En particulier, nous avons alors qu'une base de $\mathcal{H} = \frac{\left(\bm{a}^\dagger\right)^n}{\sqrt{n!}}\frac{\left(\bm{b}^\dagger\right)^m}{\sqrt{m!}}\ket{0,a}\otimes\ket{0,b} \doteq \frac{\left(\bm{a}^\dagger\right)^n}{\sqrt{n!}}\frac{\left(\bm{b}^\dagger\right)^m}{\sqrt{m!}}\ket{0,a;0,b}$.

Introduisons les opérateurs nombre $\bm{N}_a = \bm{a}^\dagger\bm{a}$ et $\bm{N}_b = \bm{b}^\dagger\bm{b}$. Ils seront utile dans la suite. En particulier, remarquons que 
\begin{equation}
  \label{eq:commutateur NaNb}
  \forall i=a,b \quad [\bm{N}_i,\bm{N}^\dagger_i] = \mathbb{I} 
\end{equation}
et que toute autre combinaison commute. 

\subsection{Spectre de l'Hamiltonien}
On introduit les opérateurs 
\begin{equation*}
  \bm{A} = \frac{1}{\sqrt{2}}\left(\bm{a}+\bm{b}\right) \quad \bm{B} = \frac{1}{\sqrt{2}}\left(\bm{a}-\bm{b}\right)
\end{equation*}
Nous voulons à présent déterminer la valeur de tous les commutateurs liant $\bm{A},\bm{B}$ et leur adjointe. Il s'agit d'une application de la relation \eqref{eq:commutateur NaNb}.
\begin{align*}
  [\bm{A},\bm{A}^\dagger] &= \frac{1}{2}\left[\left(\bm{a}+\bm{b}\right)\left(\bm{a}^\dagger+\bm{b}^\dagger\right)-\left(\bm{a}^\dagger+\bm{b}^\dagger\right)\left(\bm{a}+\bm{b}\right)\right]\\
  &= \frac{1}{2}\left[\underbrace{\left(\bm{a}\bm{a}^\dagger-\bm{a}^\dagger\bm{a}\right)}_{[\bm{a},\bm{a}^\dagger] = \mathbb{I}}+\cancelto{0}{\left(\bm{a}\bm{b}^\dagger-\bm{b}^\dagger\bm{a}\right)}+\cancelto{0}{\left(\bm{b}\bm{a}^\dagger-\bm{a}^\dagger\bm{b}\right)}+\overbrace{\left(\bm{b}\bm{b}^\dagger-\bm{b}^\dagger\bm{b}\right)}^{[\bm{b},\bm{b}^\dagger] = \mathbb{I}}\right] = 1.
\end{align*}
Le commutateur $[\bm{B},\bm{B}^\dagger]$ vaut également 1, et la preuve est identique (\emph{exercice}).
\begin{align*}
  [\bm{A},\bm{B}] &= \frac{1}{2}\left[\left(\bm{a}+\bm{b}\right)\left(\bm{a}-\bm{b}\right)-\left(\bm{a}-\bm{b}\right)\left(\bm{a}+\bm{b}\right)\right] = -\left(\bm{a}\bm{b}-\bm{b}\bm{a}\right) = -\left[\bm{a},\bm{b}\right] = 0.
\end{align*}
Les commutateurs $[\bm{A},\bm{B}^\dagger]$, $[\bm{A}^\dagger,\bm{B}]$, $[\bm{A}^\dagger,\bm{B}^\dagger]$ valent également 0, avec un développement identique (\emph{exercice}).

Ecrivons l'Hamiltonien en terme des opérateurs $\bm{A}$ et $\bm{B}$. Pour ce faire, notons que:
\begin{itemize}
  \item $\bm{a}^\dagger\bm{a} = \frac{1}{2}\left(\bm{A}+\bm{B}\right)^\dagger\left(\bm{A}+\bm{B}\right) = \frac{1}{2}\left[\bm{A}^\dagger\bm{A}+\bm{A}^\dagger\bm{B}+\bm{B}^\dagger\bm{A}+\bm{B}^\dagger\bm{B}\right]$ ;\\
  \item $\bm{b}^\dagger\bm{b} = \frac{1}{2}\left[\bm{A}^\dagger\bm{A}-\bm{A}^\dagger\bm{B}-\bm{B}^\dagger\bm{A}+\bm{B}^\dagger\bm{B}\right]$ ;
  \item $\bm{a}^\dagger\bm{b} = \frac{1}{2}\left(\bm{A}+\bm{B}\right)^\dagger\left(\bm{A}-\bm{B}\right) = \frac{1}{2}\left[\bm{A}^\dagger\bm{A}-\bm{A}^\dagger\bm{B}+\bm{B}^\dagger\bm{A}-\bm{B}^\dagger\bm{B}\right]$ ;\\
  \item $\bm{b}^\dagger\bm{a} = \frac{1}{2}\left(\bm{A}-\bm{B}\right)^\dagger\left(\bm{A}+\bm{B}\right) = \frac{1}{2}\left[\bm{A}^\dagger\bm{A}+\bm{A}^\dagger\bm{B}-\bm{B}^\dagger\bm{A}-\bm{B}^\dagger\bm{B}\right]$.
\end{itemize}
Introduisons la notation $\bm{a}^\dagger\bm{a} = \bm{N}_a$, et idem en $\bm{b}$. Il s'agit d'opérateurs nombre. Alors, l'Hamiltonien se réécrit
\begin{equation}
  H = \h\left(\omega+\lambda\right)\bm{N}_A+\h\left(\omega-\lambda\right)\bm{N}_B+\h\omega
\end{equation}
Notons que puisque $[\bm{N}_A,\bm{N}_B] = 0$, il doit exister une base orthonormée diagonalisant simultanément $\bm{N}_A$ et $\bm{N}_B$. Notons $\left\{\ket{n,A;m,B}\right\}$ une telle base. Finalement, en remarquant que l'Hamiltonien commute à la fois avec $\bm{N}_A$ et $\bm{N}_B$, nous avons que la base que nous venons de définir diagonalise l'Hamiltonien.
\begin{align*}
  \ket{n,A;m,B} &= \frac{\left(\bm{A}^\dagger\right)^n}{\sqrt{n!}}\frac{\left(\bm{B}^\dagger\right)^m}{\sqrt{m!}}\ket{0,A;0,B} &H\ket{n,A;m,B} = E_{nm}\ket{n,A;m,B}
\end{align*}
où $E_{nm} = \h\left(\omega+\lambda\right)n+\h\left(\omega-\lambda\right)m+\h\omega$. Appelons cette base $\left\{\ket{n,A;m,B}\right\}$ la \emph{base de Fock} de notre système.
\subsection{Opérateurs déplacement}
On introduit les opérateurs déplacement - tel que défini par \eqref{eq:operateur deplacement} - lié aux opérateurs d'échelle $\bm{a},\bm{b},\bm{A}$ et $\bm{B}$.
Nous voulons montrer quelques relations les liants.
\subsubsection{Quelques relations liant les opérateurs déplacement}\label{sec:quelques relations liant les operateurs deplacement}
\begin{enumerate}
  \item \emph{Lien entre $\bm{D}_a$ et $\bm{D}_A$,$\bm{D}_B$}.\label{point 1}
\begin{align*}
  \bm{D}_a(\alpha) = \exp(\alpha\bm{a}^\dagger-\alpha^*\bm{a}) &= \exp(\alpha\frac{1}{\sqrt{2}}\left(\bm{A}+\bm{B}\right)^\dagger-\alpha^*\frac{1}{\sqrt{2}}\left(\bm{A}+\bm{B}\right)) = \exp(\frac{\alpha}{\sqrt{2}}\bm{A}^\dagger-\frac{\alpha^*}{\sqrt{2}}\bm{A}+\frac{\alpha}{\sqrt{2}}\bm{B}^\dagger-\frac{\alpha^*}{\sqrt{2}}\bm{B})
\end{align*}
Notons que nous avons l'exponentielle de la somme entre $\bm{\xi}=\frac{\alpha}{\sqrt{2}}\bm{A}^\dagger-\frac{\alpha^*}{\sqrt{2}}\bm{A}$ et $\bm{\eta} = \frac{\alpha}{\sqrt{2}}\bm{B}^\dagger-\frac{\alpha^*}{\sqrt{2}}\bm{B}$. En particulier, du fait que les conditions de \ref{prop Glaubert} sont respectées\footnote{En effet, $\bm{\eta}$ et $\bm{\xi}$ commutent avec $[\bm{\eta},\bm{\xi}]$.}, nous déduisons que 
\begin{align}
  \exp(\bm{\xi}+\bm{\eta}) &= \exp(\bm{\xi})\exp(\bm{\eta})\exp(-\frac{1}{2}[\bm{\xi},\bm{\eta}]) = \bm{D}_A(\frac{\alpha}{\sqrt{2}})\bm{D}_B(\frac{\alpha}{\sqrt{2}})
\end{align}
ce qui prouve notre première relation.
  \item \emph{Lien entre $\bm{D}_b$ et $\bm{D}_A$,$\bm{D}_B$}.
\begin{align*}
  \bm{D}_b(\alpha) = \exp(\alpha\bm{b}^\dagger-\alpha^*\bm{b}) = \exp(\frac{\alpha}{\sqrt{2}}\left(\bm{A}-\bm{B}\right)^\dagger-\frac{\alpha^*}{\sqrt{2}}\left(\bm{A}-\bm{B}\right)) = \exp(\frac{\alpha}{\sqrt{2}}\bm{A}^\dagger-\frac{\alpha^*}{\sqrt{2}}\bm{A}-\frac{\alpha}{\sqrt{2}}\bm{B}^\dagger+\frac{\alpha^*}{\sqrt{2}}\bm{B})
\end{align*}
Avec un argument analogue à la discussion du point 1, nous retrouvons la formule recherchée. Dès lors, nous avons que
\begin{align}
  \bm{D}_b(\alpha) = \bm{D}_A(\frac{\alpha}{\sqrt{2}})\bm{D}_B(-\frac{\alpha}{\sqrt{2}}).
\end{align}
  \item \emph{Lien entre $\bm{D}_A$ et $\bm{D}_a$,$\bm{D}_b$}. \color{red} Rédiger la preuve.\color{black}
  \item \emph{Lien entre $\bm{D}_B$ et $\bm{D}_a$,$\bm{D}_b$}.  \color{red} Rédiger la preuve.\color{black}

En continuant sur cette lancée, nous pouvons montrer les deux relations suivantes.
\begin{align}
  \bm{D}_A(\alpha) &= \bm{D}_a(\frac{\alpha}{\sqrt{2}}) &\bm{D}_B(\alpha) = \bm{D}_a(\frac{\alpha}{\sqrt{2}})\bm{D}_b(-\frac{\alpha}{\sqrt{2}})
\end{align}
\end{enumerate}

\subsubsection{Evolution de l'oscillateur macroscopique dans un état cohérent}
On suppose ici que l'état initial du système est de la forme
\begin{equation}
  \ket{\psi_0} = \ket{\alpha,a}\ket{0,b} = \ket{\alpha,a;0,b},
\end{equation}
à savoir l'état dans lequel l'oscillateur macroscopique est dans un état cohérent. Faisons évoluer cet état.
\begin{align*}
  \ket{\psi}(t) &= \bm{U}(t)\ket{\psi_0} &\bm{U}(t) = \exp(-\frac{iHt}{\h})
\end{align*}
Or, $\ket{\alpha,a;0,b} = \bm{D}_a(\alpha\ket{0,a;0,b}) = \bm{D}_A(\frac{\alpha}{\sqrt{2}})\bm{D}_B(\frac{\alpha}{\sqrt{2}})\ket{0,a;0,b} = \ket{\frac{\alpha}{\sqrt{2}},A;\frac{\alpha}{\sqrt{2}},B}$. En effet, l'état fondamental est le même: $\ket{0,a;0,b} = \ket{0,A;0,B}$. En exploitant les résultats de la partie \ref{part 2}, nous avons directement que l'évolution de l'état initial est donnée par
\begin{equation}
  \label{eq:evolution}
  \ket{\psi}(t) = e^{-i\omega t}\ket{\frac{\alpha}{\sqrt{2}}e^{-i\left(\omega+\lambda\right)t},A;\frac{\alpha}{\sqrt{2}}e^{-i\left(\omega-\lambda\right)t},B}.
\end{equation}

\begin{definition}\label{def:systemes independants}
  Soit deux systèmes, $S_1$ et $S_2$. On dit que $S_1$ et $S_2$ sont \emph{indépendants} lorsque l'état du système $S_1\cup S_2$ peut s'écrire sous la forme $\ket{\psi_1}\otimes\ket{\psi_2}$, où $\begin{cases}
    \ket{\psi_1}\text{est l'état du système } S_1\\
    \ket{\psi_2}\text{est l'état du système } S_2
  \end{cases}$.
\end{definition}

\begin{remark}
  Si l'état de $S_1\cup S_2$ ne peut s'écrire comme le produit tensoriel des états des sous-systèmes $S_1$ et $S_2$, on dit que l'état $\ket{\psi}$ de $S_1\cup S_2$ est intriqué.
\end{remark}

Vérifions si \eqref{eq:evolution} est intriqué. Pour ce faire, réexprimons le dans la base $\ket{\alpha,a;\beta,b}$ en exploitant les relations démontrées en \ref{sec:quelques relations liant les operateurs deplacement}.

\begin{align*}
  \ket{\alpha}(t) &= e^{-i\omega t}\bm{D}_A(\frac{\alpha}{\sqrt{2}}e^{-i\left(\omega+\lambda\right)t})\bm{D}_B(\frac{\alpha}{\sqrt{2}}e^{-i\left(\omega-\lambda\right)t})\ket{0,A;0,B}\\
  &= e^{-i\omega t}\bm{D}_a(\frac{\alpha}{2}e^{-i\left(\omega+\lambda\right)t})\bm{D}_b(\frac{\alpha}{2}e^{-i\left(\omega+\lambda\right)t})\bm{D}_a(\frac{\alpha}{2}e^{-i\left(\omega-\lambda\right)t})\bm{D}_b(-\frac{\alpha}{2}e^{-i\left(\omega-\lambda\right)t})\ket{0,a;0,b}\\
  &= e^{-i\omega t}\bm{D}_a(\frac{\alpha}{2}e^{-i\left(\omega+\lambda\right)t})\bm{D}_b(\frac{\alpha}{2}e^{-i\left(\omega+\lambda\right)t})\ket{\frac{\alpha}{2}e^{-i\left(\omega-\lambda\right)t},a;-\frac{\alpha}{2}e^{-i\left(\omega-\lambda\right)t},b}
\end{align*}
On invoque à présent la relation \eqref{eq:deplacement sur etat coherent}.
\begin{align}
  \ket{\alpha}(t) &= e^{-i\omega t}\overbrace{e^{-Im\left[\frac{\norm{\alpha}^2}{4}e^{2i\lambda t}\right]}e^{-Im\left[-\frac{\norm{\alpha}^2}{4}e^{2i\lambda t}\right]}}^1\ket{\frac{\alpha}{2}\left(e^{-i\left(\omega-\lambda\right)t}+e^{-i\left(\omega+\lambda\right)t}\right),a}\otimes\ket{\frac{\alpha}{2}\left(e^{-i\left(\omega+\lambda\right)t}-e^{-i\left(\omega-\lambda\right)t}\right),b}\notag\\
  &= e^{-i\omega t}\ket{\frac{\alpha}{2}e^{-i\omega t}\left(e^{i\lambda t}+e^{-\lambda t}\right),a}\otimes\ket{\frac{\alpha}{2}e^{-i\omega t}\left(e^{-i\lambda t}-e^{i\lambda t}\right),b} = e^{-i\omega t}\ket{\alpha e^{-i\omega t}\cos(\lambda t),a}\otimes\ket{\color{red}-\color{black}i\alpha e^{-i\omega t}\sin(\lambda t),b}\label{eq:etat naif evolue}
\end{align}
On voit donc que l'état $\ket{\alpha}(t)$ n'est \textbf{pas} intriqué. Cela semble correspondre à notre expérience quotidienne: une particule en Haute-Savoie n'influe pas sur un.e physicien.ne en Normandie - les deux sont, d'après notre vécu, indépendant (en le sens de la définition \ref{def:systemes independants}).

\subsection{Mise en évidence de la décohérence quantique}
Nous supposons à présent que l'état initial est de la forme
\begin{equation}
  \ket{\psi_0} = \mathcal{N}\left(\ket{\alpha,a}+\ket{\beta,a}\right)\otimes\ket{0,b}\label{eq:quantum decoherence initial state}
\end{equation}
En particulier, nous avons alors que l'état initial est une superposition d'oscillateurs harmonique macroscopiques distincts. De manière générale, deux oscillateurs sont macroscopiquement distinguables \emph{si et seulement si} $\alpha\neq\beta$.
\subsubsection{Superposition d'états classiques}
Considérons en particulier deux pendules pesants classique en \emph{opposition de phase}. Utilisons les résultats \eqref{mean position} et \eqref{mean P}, lesquels permettent d'imiter les oscillateurs harmonique classique tout en restant dans le formalisme quantique des états cohérents que nous traitons ici. En particulier,
\begin{align*}
  \expval{\bm{X}}_\alpha (t) &= \sqrt{\frac{2\h}{m\omega}}\norm{\alpha}\cos(\omega t) &\expval{\bm{X}}_\beta (t) = \sqrt{\frac{2\h}{m\omega}}\norm{\beta}\cos(\omega t+\pi)\\
  \expval{\bm{P}}_\alpha (t) &= -\sqrt{2m\h\omega}\norm{\alpha}\sin(\omega t) &\expval{\bm{P}}_\beta (t) = -\sqrt{2m\h\omega}\norm{\beta}\sin(\omega t+\pi)
\end{align*}
Cela nous indique que $\norm{\alpha} = -\norm{\beta}$. La condition générale que nous avons défini étant respectée, deux oscillateurs macroscopiques en opposition de phase sont bien macroscopiquement distinguables. En particulier, l'état initial que nous avions défini se réécrit
\begin{align*}
  \ket{\psi_0} = \mathcal{N}\left(\ket{\alpha,a}+\ket{-\alpha,a}\right)\otimes\ket{0,b}
\end{align*}
Il ne nous reste plus qu'à déterminer $\mathcal{N}$. Pour ce faire, notons que \ref{prop:condition de normalisation anal} se réécrit algébriquement comme suit.
\begin{property}
  La condition de normalisation de tout état $\ket{\psi}$ impose que $\braket{\psi} = 1$.
\end{property}

Il ne nous reste plus qu'à l'appliquer. Rappelons également la résultat $\eqref{15}$, nous donnant l'expresison du produit scalaire $\braket{\alpha}{\beta}$ entre deux états cohérents.
\begin{align*}
  \braket{\psi_0} &= \mathcal{N}^2\left(\bra{\alpha,a}+\bra{-\alpha,a}\right)\left(\ket{\alpha,a}+\ket{-\alpha,a}\right) = \mathcal{N}^2\left(2+2e^{-2\norm{\alpha}^2}\right)
\end{align*}

Un oscillateur est caractérisé par un nombre moyen de particules très grand (de l'ordre de $10^{30}$)\footnote{cf. énoncé.}. En vertue de la relation \eqref{eq:mean N}, $\norm{\alpha}^2 \approx 10^{30}$. Ainsi, 
\begin{align}
  \mathcal{N}^2 &= \frac{1}{2} &\mathcal{N} = \frac{1}{\sqrt{2}}
\end{align} 
à une phase (globale) près.

\subsubsection{Evolution et temps de décohérence}

Nous voulons déterminer $\ket{\psi}(t)$. Pour ce faire, nous appliquons l'opérateur d'évolution à \eqref{eq:quantum decoherence initial state}, comme suit:
\begin{align*}
  \ket{\psi}(t) &= \bm{U}(t)\mathcal{N}\left(\ket{\alpha,a}+\ket{\beta,a}\right)\otimes\ket{0,b} = \mathcal{N}\left(\bm{U}(t)\ket{\alpha,a}+\bm{U}(t)\ket{\beta,a}\right)\otimes\ket{0,b}
\end{align*}
En distribuant le produit tensoriel, on observe qu'il s'agit exactement de \eqref{eq:etat naif evolue}. L'état évolué est donc donné par la relation
\begin{align}
  \ket{\psi}(t) &= e^{-i\omega t}\mathcal{N}\left(\ket{\alpha e^{-i\omega t}\cos(\lambda t),a;i\alpha e^{-i\omega t}\sin(\lambda t),b}+\ket{\beta e^{-i\omega t}\cos(\lambda t),a;i\beta e^{-i\omega t}\sin(\lambda t),b}\right)
\end{align}

Cela nous ammène à déterminer le \emph{temps de décohérence}.
\begin{definition}
  Le temps de décohérence $t_D$ d'un système quantique est le temps nécessaire pour que les états de l'environnement deviennent proche de zéro.\label{def:temps decoherence} 
\end{definition}

Dans le cadre de notre modèle, l'environnement correspond au second oscillateur harmonique, de même fréquence que l'oscillateur macroscopique. Ainsi, cela revient à résoudre
\begin{align}
  \braket{i\alpha e^{i\omega t_D}\sin(\lambda t_D),b}{i\beta e^{i\omega t_D}\sin(\lambda t_D)} &= e^{-1}\notag\\
  \exp(-\frac{1}{2}\left(\norm{\alpha\sin(\lambda t_D)}^2+\norm{\beta\sin(\lambda t_D)}^2+2\alpha\alpha^*\sin(\lambda t_D)^2\right)) &= e^{-1}\notag\\
  \left(\norm{\alpha}^2+\norm{\beta}^2-2\alpha\beta^*\right)\sin(\lambda t_D)^2 &= 2\notag\\
  t_D = \frac{1}{\lambda}\arcsin(\sqrt{\frac{2}{\norm{\alpha}^2+\norm{\beta}^2-2\alpha\beta^*}})&\label{eq:decoherence time general}
\end{align}

Revenons sur notre modèle jouet avec deux pendules simples en opposition de phase. Dans ce modèle, nous avons vu que $\norm{\alpha}^2\approx 10^{30}\approx\norm{\beta}^2$. En particulier, il s'ensuit que l'argument de la fonction $\arcsin(x)$ est $x<<1$, tout en restant strictement positif. En particulier, à cette échelle là, nous pouvons admettre $\arcsin(x)\approx x$. Nous avons alors, avec $\norm{\alpha}=-\norm{\beta}$, $t_D \approx \frac{1}{\lambda}\sqrt{\frac{1}{4\norm{\alpha}^2}}<<1s$.

Le temps de relaxation d'un pendule simple est sa période: $t_R=T=2\pi\omega=2\pi\sqrt{\frac{g}{L}}\approx \frac{2\pi}{9}>1s$, en invoquant la première relation dans \eqref{eq:46}.

Nous avons donc bien que le temps de décohérence - c'est à dire, selon la définition \ref{def:temps decoherence} - le temps nécessaire pour que les états superposés tendent vers 0 - est petit devant la période d'un pendule simple. Il n'est dès lors pas étonant que nous n'observions pas de superposition d'états dans notre quotidien. Par exemple, il est impossible d'observer un chat à la fois mort et vivant. Miaou! {\DejaSans 😺}

\paragraph{Défaut du modèle}

Les fonctions trigonométriques sont périodiques: la relation \eqref{eq:decoherence time general} donne alors une périodicité au temps de décohérence $t_D$. Cela signifirait que nous pourrions observer des états superposés par interval régulier: cela n'étant empiriquement pas le cas, nous avons une \emph{incohérence} (haha) dans notre modèle. Une solution à ce problème, mis en évidence dans l'énoncé, serait de modéliser l'environnement non pas par un oscillateur harmonique à une fréquence fixe, mais par un ensemble infini d'oscillateurs ayant une distribution des fréquences.

\subsubsection{Décohérence - ou comment se passer du postulat de réduction du paquet d'onde}

Historiquement, l'idée de la décohérence quantique fût proposée comme alternative au principe de \emph{réduction du paquet d'onde}. Ce dernier postule qu'immédiatement après une mesure, l'état du système est projeté sur le sous-espace propre associé au résultat mesuré. On voit le processus de mesure comme une intéraction complexe entre le \emph{dispositif expérimental}, l'\emph{environnement} et le \emph{système}.\\


\end{document}